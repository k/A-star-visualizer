\documentclass[11pt,draft]{article}

 
\usepackage{hyperref}
\hypersetup{
    colorlinks=true,
    linkcolor=blue,
    filecolor=magenta,      
    urlcolor=cyan,
    pdftitle={Kenneth Bambridge --- Assignment 1},
    bookmarks=true,
    pdfpagemode=FullScreen,
}

\usepackage{amsmath}
\usepackage{amssymb}
% packages that allow mathematical formatting

\usepackage{graphicx}
% package that allows you to include graphics

% \usepackage{setspace}
% package that allows you to change spacing

% \onehalfspacing
% text become 1.5 spaced

\usepackage{fullpage}
% package that specifies normal margins

\begin{document}

\title{Intro to AI Assignment 1 --- Heuristic Search}
\author{Kenneth Bambridge --- kmb394}
\date{\today}
\maketitle
\tableofcontents

\section{Introduction}
% Describe the purpose of this project and hint at what was learned

\section{Algorithms}
% Describe the basic theory behind the algorithms and heuristics, answers questions related to them

\subsection{Uniform Cost Search}

\subsection{A*}

\subsection{Weighted A*}

\subsection{Sequential A*}

\subsubsection{Proof that Sequential A* is $w_1$ subomptimal}
Given that the $g$ value of any state $s$ expanded by the Weighted A* algorithm is at most $w_1$ suboptimal for an admissible and consistent heuristic, consider a state $s_i$ in the $OPEN_0$ queue that is on the least cost path to $s_{goal}$. 
During the initialization loop starting at line 13, all of the $OPEN_i$ fringes are populated with the start node, which is trivially on the shortest path to $s_{goal}$.
For the iterations starting at line 19, the only time the $OPEN_0$ data structure is inserted or updated is on line 11, where it is inserted with the value $g_0(s) + w_1*h_0(s)$ for all neighbors where their $g$ values are less then the existing keys in the fringe.
\\\\
Since during every \verb_ExpandState_ call, all neighbors are considered for the node that was originally on the shortest path (the start node), there must also be a node $s_{n}$ considered every time $OPEN_0$ is updated that is on the shortest path to $s_{goal}$. This node's $g$ value is also updated if it is less then the existing value. The key of this value is $g(s_n) + w_1*h(s_n)$. The heuristic is admissible, so $h(s_n) \leq c^*(s_n) \implies w1*h(s_n) \leq w1*c^*(s_n)$. Since $s_{n}$ is on the shortest path to goal, $g(s_n)$ represents the shortest path from $s_{start} \rightarrow s_n$, therefore $g(s_n) + w_1*h(s_n) \leq w_1*c*(s_n)$.
Since this `minimum' node exists for the initialization and is maintained in every loop, it exists for the duration of the algorithm.
\\\\
The heuristic is admissible and consistent, therefore $s_n$ must also be the minimum key in $OPEN_0$ because 

\subsubsection{Proof that Sequential A* is $w_1*w_2$ subomptimal}
% Then, prove that when the Sequential Heuristic A* terminates in the ith search, that gi(sgoal) < w1*w2*c*(sgoal)    
The program can exit in one of two ways, either with the anchor search or via an non-anchor heuristic.
If all the non-anchor heuristics have minimum keys greater than the anchor, the anchor will be run and the goal returned if $g_0(s_{goal})$ is less then the minimum key in $OPEN_0$ and infinity. In this case the output is $w_1$-suboptimal as proven above. \\
When the program exits with an non-anchor heuristic, it is run when the minimum key is less then $w_2*OPEN_0.\text{Minkey} < w_2*w_1*c^*(s_goal)$.
Therefore when the algorithm exits in this way the solutions is $w_1*w_2$-suboptimal.

\subsection{Integrated A*}
\subsubsection{Proof i}
\paragraph{No state expanded more than twice.}
\begin{enumerate}
    \item When a state is expanded, it is inserted into one of the CLOSED data structures. (a) In the inadmissible branch, it is inserted into $CLOSED_{inad}$ (line 36). (b) In the admissible branch, it is inserted into $CLOSED_{anchor}$ (line 44).
    \item A state is never inserted into $OPEN_i, \forall i$ while it is in the $CLOSED_{anchor}$ data structure (line 12).
    \item A state is never inserted into $OPEN_i, \forall i \neq 0$ while it is in the $CLOSED_{inad}$ data structure (line 14).
    \item A state can only be expanded after being inserted into an $OPEN_i$ data structure (line 29, 34, 42).
    \item When a state is expanded, the state is removed from $OPEN_i, \forall i$ (line 4).
    \item By combining (1a), (2), (4), and (5), a state will be expanded at most once in the inadmissible branch (line 35).
    \item By combining (1b), (2), (4), and (5), a state will be expanded at most once in the anchor branch (line 43).
\end{enumerate}
 Since a non-start state will only be expanded at most once in each branch, therefore no state is expanded more than twice other than the start node.

\subsubsection{Proof ii}
\paragraph{State expanded in the anchor search is never re-expanded.}

\begin{enumerate}
    \item When the anchor search expands a state, the state is added to $CLOSED_{anchor}$ (line 44).
    \item Only the anchor search branch can expand nodes in $OPEN_0$ (line 42).
    \item The anchor search branch only expands nodes in $OPEN_0$ (lines 42--44).
    \item A state is never inserted into $OPEN_i, \forall i$ while it is in the $CLOSED_{anchor}$ data structure (line 12).
    \item After a state is expanded in the anchor search, it is not in $OPEN_i, \forall i$ (line 4).
    \item A state can only be expanded while in an $OPEN_i$ data structure (line 29, 34, 42).
    \item Therefore, a state expanded in the anchor search is never re-expanded.
\end{enumerate}

\subsubsection{Proof iii}
State expanded in an inadmissible search can only be re-expanded in the anchor search if its g-value is lowered

\begin{enumerate}
    \item A state expanded in an inadmissible can add the the state to $OPEN_0$ (line 13).
    \item Whenever a key is added to an inadmissible $OPEN_i$, then it's key must be less then or equal to the corresponding admissible heuristic (line 16)
    \item If a state $s_i$ was expanded in an inadmissible heuristic, then it was removed from $OPEN_0$ (line 4).
    \item The $g$ value is retained, and then is only re-added if the check in line 10 is satisfied for $n \in \text{succ}(s_i)$, which for which it must be a lower g value.
\end{enumerate}

\section{Heuristics}

\subsection{Chebychev Distance}

\subsection{Manhattan Distance}

\subsection{Diagonal Distance}

\subsection{Euclidian Distance}


\section{Implementation Notes}

\subsection{Map Generation}
% Discuss how the map was generated, what was the data structure, what tools were used

\subsection{Algorithm Implementation}
% Discuss the choices for implementing the algorithms the way we did
% Talk about optimizations, memoization, heap/bisect

\subsection{GUI Visualizer}
% Discuss how the GUI visualizer is implemented, what tools were used, etc.

\section{Benchmarks}

\subsection{Overall Comparison}
\subsection{Comparison of Heuristic Modifiers}
\subsection{Uniform Cost Search}
\subsection{A*}
\subsection{Weighted A*}
\subsection{Sequential A*}
\subsection{Integrated A*}
\subsection{Conclusion}


\end{document}
